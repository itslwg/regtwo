% Created 2020-03-11 Wed 17:04
% Intended LaTeX compiler: pdflatex
\documentclass[11pt]{article}
\usepackage[utf8]{inputenc}
\usepackage[T1]{fontenc}
\usepackage{graphicx}
\usepackage{grffile}
\usepackage{longtable}
\usepackage{wrapfig}
\usepackage{rotating}
\usepackage[normalem]{ulem}
\usepackage{amsmath}
\usepackage{textcomp}
\usepackage{amssymb}
\usepackage{capt-of}
\usepackage{hyperref}
\usepackage{minted}
\usepackage[margin=1.25in]{geometry} \usepackage{booktabs} \usepackage{graphicx} \usepackage{adjustbox} \usepackage{amsmath} \usepackage{amsthm} \newtheorem{definition}{Definition} \usepackage{bookmark} \usepackage{tabularx}
\author{Ludde}
\date{\today}
\title{}
\hypersetup{
 pdfauthor={Ludde},
 pdftitle={},
 pdfkeywords={},
 pdfsubject={},
 pdfcreator={Emacs 26.2 (Org mode 9.1.14)}, 
 pdflang={English}}
\begin{document}

\begin{titlepage}
\centering
\includegraphics[width=0.15\textwidth]{example-image-1x1}\par\vspace{1cm}
{\scshape\LARGE Kungliga Tekniska Högskolan \par}
\vspace{1cm}
{\scshape\Large SF2930 Regression Analysis \par}
\vspace{1.5cm}
{\huge\bfseries Report II \\  \par}
\vspace{2cm}
{\Large\itshape Isac Karlsson \\ Ludvig Wärnberg Gerdin}
\vfill
Examiner \par
\textsc{Tatjana Pavlenko}

\vfill

{\large \today\par}
\end{titlepage}

\newpage
\tableofcontents
\newpage

\section{Introduction}
\label{sec:org136b02b}
\subsection{Background}
\label{sec:org13e0b17}
Most of the tractors in Sweden have a third party liability insurance, because they are required by law. 
In southern Europe a few large players have dominated the sales of tractor insurances. Our main task this
project is to create our own tractor tariff. The price \(p_i\) for tariff cell \(i\) is defined as

\begin{equation}
\label{eq:orgf83f1f4}
  p_i = \gamma_0 \prod_{k = 1}^M \gamma_{k,j}   
\end{equation}

Here \(\gamma_{\text{0}}\) corresponds to the base level and \(\gamma_{k,j}\) are the risk factors for variable \(k\) and 
variable group \(j\). For example, letting the variable \texttt{VehicleAge} correspond to variable 
number \(k = 1\) and letting a particular variable group \(\texttt{VehicleAge} < 2\) correspond to \(j = 2\), 
then the risk factor for the \texttt{VehicleAge} of a tractor that is 1 year old would be \(\gamma_{12}\).

\subsection{Data}
\label{sec:org94bf999}

\emph{If P\&C} have provided us with data to train this price model, example given in the Table 1.


\begin{table}
\caption{Data example}
\centering
\adjustbox{max width=\linewidth}{
\begin{tabular}{rrrllrrr}
\toprule
RiskYear & VehicleAge & Weight & Climate & ActivityCode & Duration & NoOfClaims & ClaimCost\\
\midrule
2010 & 009 & 3830 & North & Construction & 0.63 & 1 & 627099\\
2008 & 001 & 400 & South & Missing & 0.59 & 1 & 253850\\
\vdots & \vdots & \vdots & \vdots & \vdots & \vdots & \vdots & \vdots\\
\bottomrule
\end{tabular}
\end{table}

\subsection{Problem description}
\label{sec:org203659f}
\subsubsection{Risk Differentiation and Grouping}
\label{sec:orgbefb16a}

Using GLM analysis we aim to make each group “Risk homogeneous” and that they contain enough data to
get a stable GLM analysis, meanwhile handling imperfections in the dataset.

\subsubsection{Levelling}
\label{sec:org995a514}

Here we aim to calculate \(\gamma_0\) such that the forecasted claim costs for each insurance are covered by the
the price for each insurance on a full year basis. We use a ratio between the estimated claim cost and
the total premium of 90\%. Lastly we calculate the base level \(\gamma_0\) from the formula given in (\ref{eq:orgf83f1f4})

\section{Methods and Methodological Considerations}
\label{sec:orgce3c068}
\subsection{Grouping and Risk Differentiation}
\label{sec:orgc57fe0b}

The criteria on which we based our grouping was

\begin{enumerate}
\item Each group should be risk homogeneous, and
\item Each group should have enough data to make the GLM estimates stable.
\end{enumerate}
Greater emphasis were placed on fulfilling criteria 2) due to it being more concrete. In order to do that
we choose cut-offs that placed a fairly equal shares of data in each risk group. 

The resulting cut-offs and risk groups are found in section \ref{sec:orgc277d6e}.

\subsection{Levelling}
\label{sec:orgb763a92}

From the results of section \ref{sec:orgc57fe0b}, we get risk factor estimates for each
level of each predictor. These are henceforth referred to as the "group factors".

For each corresponding sub-assignment presented under Levelling in the project description, we conducted the
following:

\begin{enumerate}
\item From the original data, we selected those rows (or tractors) that had a \texttt{RiskYear} 2016. That 
way the GLM analysis were only conducted on the active customers, leaving out those that weren't 
customers to If in 2016. 

Following the GLM-script each row were aggregated to tariff cells.
From the aggregated data we calculated the expected yearly claim-cost per tariff cell by dividing the
claim cost by the duration for each row. The rationale for this was to enable us to analyse the 
yearly cost, even if the insurances had not been active for all of 2016. 

The estimated claim cost for the coming year is then simply the sum of all of the estimated yearly 
claim costs for each tariff cell.

\item Recall first that the \emph{If P\&C} has a ratio target of 0.9 between the estimated claim cost and the total premium.
Using that the total premium is \(P = \sum_i p_i\), where \(p_i\) is the premium or price for tariff cell \(i\) defined 
in (\ref{eq:orgf83f1f4}), and the total expected yearly cost is \(C = \sum_i c_i\), where \(c_i\) is the expected yearly cost 
for tariff cell \(i\), we get

\begin{equation}
\label{eq:org5676fb9}
  \frac{C}{P} = 0.9 
\end{equation}

Using that we have an estimated value for the total expected claim cost, we reorder the equation to get

\begin{equation}
\label{eq:org8bb025c}
\frac{C}{P} = 0.9 \iff \frac{C}{0.9} = P
\end{equation}

\item The formula for the price of tariff cell \(i\) is defined in (\ref{eq:orgf83f1f4}). As a reminder, this was

\[
      p_i = \gamma_0 \prod_{k = 1}^M \gamma_{k,j}
      \]

Inserting (\ref{eq:orgf83f1f4}) into the formula for the total premium we get

\begin{equation}
\label{eq:org4a58580}
P = \sum_i p_i = \sum_i \bigg (\gamma_0 \prod_{k = 1}^M \gamma_{j,k} \bigg)_i =  \gamma_0 \sum_i \bigg ( \prod_{k = 1}^M \gamma_{j,k} \bigg)
\end{equation}
and subsequently inserting (\ref{eq:org4a58580}) into (\ref{eq:org8bb025c}) and solving for \(\gamma_0\) 

\begin{equation}
\label{eq:org879f97e}
\frac{C}{0.9} = \gamma_0 \sum_i \bigg ( \prod_{k = 1}^M \gamma_{j,k} \bigg) \implies \frac{C}{\sum_i \bigg ( \prod_{k = 1}^M \gamma_{j,k} \bigg)_i} = \gamma_0
\end{equation}

In other words: Row by row in the aggregated data, we map the row characteristics to the
corresponding risk factor  in the group factor table, and subsequently take the product of all 
the risk factors to obtain the total risk factor for that tariff cell. By (\ref{eq:org879f97e}), we obtain the 
base level \(\gamma_0\) by dividing the total expected cost by the sum of all total risk factors.
\end{enumerate}

\section{Results}
\label{sec:orgc277d6e}

We adjusted the data slightly to adjust for some odd rows. The rows with \(\texttt{Duration} = 0\) were removed. 
Since we are interested in the tractors that has been exposed to risk, and the tractors where 
\(\texttt{Duration} = 0\) make up a small fraction of the total data, we decided to remove these rows. 

We noted that some rows had suspiciously low values for the \texttt{Weight} predictor, e.g. a weight of 0. 
Since the fraction of rows with \(\texttt{Weight} = 0\) were small (0.02\%), removing the rows would not have
a large impact on the results. However, since many 
of these rows correspond to a particular ActivityCode (namely Middle H - Hotels and restaurants), 
we believe that we are missing the contextual dimensions needed to decide whether these should be counted 
as wrong inputs in the dataset. In the end we decided to leave them in the data as a part of the
group < 1000. An alternative would have been to include those rows as a separate level, however in that case the risk factor
corresponding to this level would have been inflated (since a claim on a small duration on would have resulted
in a predicted claim frequency), which is undesirable.

In the exploratory analysis of the data, we identified the use of "Other" as a factor level to \texttt{ActivityCode}. 
We assume that this level can be mapped to more specific types of businesses internally by If. In a future version
of this model, the model could input more granular groups of \texttt{ActivityCode} to potentially improve 
the performance of the model.

The variable groups and corresponding estimated risk factors are presented in Table \ref{tab:risk_groups}.
For each sub-question in the Levelling assignment we got the following results

\begin{enumerate}
\item The total expected cost for 2017 would be 170033.9 kr.
\item The total premium would be 188926.6 kr, and
\item Mapping the risk factors to each respective tariff cell and calculating \(\gamma_0\) we got

\[
     \gamma_0 = 238.8046
     \]
\end{enumerate}

In order to evaluate our model we used the Akaike Information Criterion (AIC). Considering e.g. 
the qualitative importance of the age of the insured tractors when estimating the claim severity, we decided to test 
whether leaving out this predictor would reduce the predictive performance of the model. The results 
of this evaluation are presented in Table \ref{tab:performance_table}. The AIC was lower for the both models 
when keeping the \texttt{VehicleAge\_group} predictor, hence we conclude that this predictor should be included in
the model.

A more exhaustive and thorough way of evaluating the model would have been to run all-possible regression 
with AIC in order to evaluate the importance of the other predictors. This, however, is left 
for a future analysis.

\input{../risk_groups.tex}

\input{../performance_table.tex}

\section{Conclusion}
\label{sec:org4776bc7}

For each sub-question in the Levelling assignment we found that

\begin{enumerate}
\item The total expected cost for 2017 would be 170033.9 kr.
\item The total premium would be 188926.6 kr, and
\item \(\gamma_0 = 238.8046\)
\end{enumerate}
\end{document}