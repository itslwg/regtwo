% Created 2020-03-09 Mon 16:03
% Intended LaTeX compiler: pdflatex
\documentclass[11pt]{article}
\usepackage[utf8]{inputenc}
\usepackage[T1]{fontenc}
\usepackage{graphicx}
\usepackage{grffile}
\usepackage{longtable}
\usepackage{wrapfig}
\usepackage{rotating}
\usepackage[normalem]{ulem}
\usepackage{amsmath}
\usepackage{textcomp}
\usepackage{amssymb}
\usepackage{capt-of}
\usepackage{hyperref}
\usepackage{minted}
\usepackage[margin=1.25in]{geometry} \usepackage{booktabs} \usepackage{graphicx} \usepackage{adjustbox} \usepackage{amsmath} \usepackage{amsthm} \newtheorem{definition}{Definition} \usepackage{bookmark}
\author{Ludde}
\date{\today}
\title{}
\hypersetup{
 pdfauthor={Ludde},
 pdftitle={},
 pdfkeywords={},
 pdfsubject={},
 pdfcreator={Emacs 26.2 (Org mode 9.1.14)}, 
 pdflang={English}}
\begin{document}

\begin{titlepage}
\centering
\includegraphics[width=0.15\textwidth]{example-image-1x1}\par\vspace{1cm}
{\scshape\LARGE Kungliga Tekniska Högskolan \par}
\vspace{1cm}
{\scshape\Large SF2930 Regression Analysis \par}
\vspace{1.5cm}
{\huge\bfseries Report II \\  \par}
\vspace{2cm}
{\Large\itshape Isac Karlsson \\ Ludvig Wärnberg Gerdin}
\vfill
Examiner \par
\textsc{Tatjana Pavlenko}

\vfill

{\large \today\par}
\end{titlepage}

\newpage
\tableofcontents
\newpage

\section{Introduction}
\label{sec:org98daf89}
\subsection{Background}
\label{sec:orga14df41}
Most of the tractors in Sweden have a third party liability insurance, because they are required by law. 
In southern Europe a few large players have dominated the sales of tractor insurances. Our main task this
project is to create our own tractor tariff on the form:

\begin{equation}
  \text{Price} = \gamma_0 \prod_{k = 1}^M \gamma_{k,i}  
\end{equation}

Here y0 corresponds to the base level and yki are the risk factors and corresponds to each individual 
tractor. 

\subsection{Data}
\label{sec:org92878bd}

If P\&C have provided us with data to train this price model, example given in the table below.

\subsection{Problem description}
\label{sec:org0d20bb8}
\subsubsection{Risk Differentiation and Grouping}
\label{sec:orgff1afec}

Using GLM analysis we aim to make each group “Risk homogeneous” and that they contain enough data to
get a stable GLM analysis, meanwhile handling imperfections in the dataset.

\subsubsection{Levelling}
\label{sec:org2434c16}

Here we aim to calculate yo such that the forecasted claim costs for each insurance are covered by the
the price for each insurance, on a full year basis. We us a ratio between the estimated claim cost and
the total premium of 90\%. Lastly we calculate the base level yo from the formula given in (1).

\section{Methods}
\label{sec:orge528f3b}
\subsection{Grouping and Risk Differentiation}
\label{sec:org6ae7275}

The criteria on which we based our groups were that

\begin{enumerate}
\item Each group should be risk homogeneous, and
\item Each group should have enough data to make the GLM estimates stable.
\end{enumerate}
Greater emphasis were placed on fulfilling criteria 2) due to it being more concrete.
In order to do that we considered cut-offs that placed a fairly equal shares of data in 
each risk group. 

The resulting cut-offs and risk groups are found in section \ref{sec:org898dae6}.

\subsubsection{Levelling}
\label{sec:org1383086}

Firstly, we subset the data rows (or customers) that had a \texttt{RiskYear} 2016. That 
way the GLM analysis were only conducted on the active customers. not those that weren't
customers to If anymore. From the aggregated data, initialized as \texttt{glmdata2} in the
provided script, we calculated the yearly claim-cost per tractor by dividing the summed duration
in each duration by the corresponding summed claim cost in


In pseudo-code, the process was 

\begin{minted}[]{r}
    for each 
\end{minted}
\section{Results}
\label{sec:org898dae6}
\section{Conclusion}
\label{sec:org13e31d1}
\end{document}