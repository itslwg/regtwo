% Created 2020-03-10 Tue 13:46
% Intended LaTeX compiler: pdflatex
\documentclass[11pt]{article}
\usepackage[utf8]{inputenc}
\usepackage[T1]{fontenc}
\usepackage{graphicx}
\usepackage{grffile}
\usepackage{longtable}
\usepackage{wrapfig}
\usepackage{rotating}
\usepackage[normalem]{ulem}
\usepackage{amsmath}
\usepackage{textcomp}
\usepackage{amssymb}
\usepackage{capt-of}
\usepackage{hyperref}
\usepackage{minted}
\usepackage[margin=1.25in]{geometry} \usepackage{booktabs} \usepackage{graphicx} \usepackage{adjustbox} \usepackage{amsmath} \usepackage{amsthm} \newtheorem{definition}{Definition} \usepackage{bookmark}
\author{Ludde}
\date{\today}
\title{}
\hypersetup{
 pdfauthor={Ludde},
 pdftitle={},
 pdfkeywords={},
 pdfsubject={},
 pdfcreator={Emacs 26.2 (Org mode 9.1.14)}, 
 pdflang={English}}
\begin{document}

\begin{titlepage}
\centering
\includegraphics[width=0.15\textwidth]{example-image-1x1}\par\vspace{1cm}
{\scshape\LARGE Kungliga Tekniska Högskolan \par}
\vspace{1cm}
{\scshape\Large SF2930 Regression Analysis \par}
\vspace{1.5cm}
{\huge\bfseries Report II \\  \par}
\vspace{2cm}
{\Large\itshape Isac Karlsson \\ Ludvig Wärnberg Gerdin}
\vfill
Examiner \par
\textsc{Tatjana Pavlenko}

\vfill

{\large \today\par}
\end{titlepage}

\newpage
\tableofcontents
\newpage

\section{Introduction}
\label{sec:org5b81261}
\subsection{Background}
\label{sec:org66de239}
Most of the tractors in Sweden have a third party liability insurance, because they are required by law. 
In southern Europe a few large players have dominated the sales of tractor insurances. Our main task this
project is to create our own tractor tariff on the form:

\begin{equation}
  \text{Price} = \gamma_0 \prod_{k = 1}^M \gamma_{k,i}  
\end{equation}

Here y0 corresponds to the base level and yki are the risk factors and corresponds to each individual 
tractor. 

\subsection{Data}
\label{sec:org821dc54}

If P\&C have provided us with data to train this price model, example given in the table below.

\subsection{Problem description}
\label{sec:org408fb19}
\subsubsection{Risk Differentiation and Grouping}
\label{sec:orgd4d9168}

Using GLM analysis we aim to make each group “Risk homogeneous” and that they contain enough data to
get a stable GLM analysis, meanwhile handling imperfections in the dataset.

\subsubsection{Levelling}
\label{sec:org00ad7b6}

Here we aim to calculate yo such that the forecasted claim costs for each insurance are covered by the
the price for each insurance, on a full year basis. We us a ratio between the estimated claim cost and
the total premium of 90\%. Lastly we calculate the base level yo from the formula given in (1).

\section{Methods and Methodological Considerations}
\label{sec:org12395d3}
\subsection{Grouping and Risk Differentiation}
\label{sec:orgc31a25a}

The criteria on which we based our groups were that

\begin{enumerate}
\item Each group should be risk homogeneous, and
\item Each group should have enough data to make the GLM estimates stable.
\end{enumerate}
Greater emphasis were placed on fulfilling criteria 2) due to it being more concrete. In order to do that
we considered cut-offs that placed a fairly equal shares of data in each risk group. 

In addition to the above, we adjusted the data slightly to adjust for some odd rows. For example, 
the tractors with a weight of 0 were placed in a category of its own. Also, we noticed the use 
of (Other) and Other as different factor levels for the ActivityCode regressor. These levels, however,
were left as they are. The rationale was that we assume a tractor with ActivityCode = Other and
ActivityCode = (Other) are mapped to more specific types of businesses internally by If. In a future version
of this model, the model could input more granular groups of ActivityCode to potentially improve 
the performance of the model.

The resulting cut-offs and risk groups are found in section \ref{sec:org73a0f29}.

\subsection{Levelling}
\label{sec:org82717ed}

From the results of section \ref{sec:orgc31a25a}, we get risk factor estimates for each
level of each predictor. These are henceforth referred to as the "group factors".

For each corresponding assignment presented under Levelling in the project description, we conducted the
following:

\begin{enumerate}
\item From the original data, we selected those rows (or tractors) that had a \texttt{RiskYear} 2016. That 
way the GLM analysis were only conducted on the active customers, leaving out those that weren't 
customers to If anymore.

Following the GLM-script each row were aggregated to include one row per combination of variables. 
From the aggregated data, we calculated the expected yearly claim-cost per tractor by dividing the
duration by the corresponding claim cost for each row. The rationale was to enable us to analyse the 
yearly cost, even if the insurances had not been active for all of 2016.

\item 
\end{enumerate}


In pseudo-code, the process was 

\begin{minted}[]{r}
   for each 
\end{minted}

\section{Results}
\label{sec:org73a0f29}
\subsection{Grouping and Risk Differentiation}
\label{sec:org1c88ca1}

\subsection{Levelling}
\label{sec:orga0f5b00}
\input{../performance_table.tex}

\input{../risk_groups.tex}

\section{Conclusion}
\label{sec:orgc79e91f}
\end{document}